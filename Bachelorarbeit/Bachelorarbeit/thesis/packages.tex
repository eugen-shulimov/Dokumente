\usepackage{amsmath, amssymb, amsfonts}% mathematical symbols and the like
\usepackage{amsthm}% definitions, theorems, etc.
\usepackage[colorinlistoftodos]{todonotes}% marking open todos in text/on margins
\usepackage{subfig}% multi-part figures with separate captions per part
\usepackage{url}% render URLs correctly and make them clickable through the hyperref package
\usepackage{longtable}% tables that span multiple pages
\usepackage{booktabs}% tables that actually look good
\usepackage[nolist]{acronym}% consistently use acronyms

\usetikzlibrary{arrows.meta}


\usepackage{listings}
\usepackage{color}

\usepackage{float}
\usepackage[absolute]{textpos}
\usepackage[ruled,linesnumbered]{algorithm2e}
\usetikzlibrary{shapes,snakes}


\usepackage{setspace}


%\usepackage{standalone}
\usepackage[numbers]{natbib}
\bibliographystyle{abbrvnat}






\usepackage{amsmath}
\usepackage{amsfonts}
\usepackage{amssymb}
\usepackage{graphicx}
\usepackage{float} %%% for figure floating argument [H]
\usepackage{tikz}
\usepackage{capt-of}
\usepackage{setspace}



%%% Use a box vertically centered.
%%% Optional Argument: Scaling factor of the box, default is one
%%% Second Argument: box name
\newcommand{\useboxVerticallyCentered}[2][1]{\raisebox{-.45\baselineskip}{\scalebox{#1}{\usebox{#2}}}}

%%% Includes graphics from a file (path to that file is the first argument) within a sentence, scaled to the text size.
\newcommand{\includegraphicsVCentered}[2]{\ensuremath{\vcenter{\hbox{\includegraphics[#1]{#2}}}}}
\newcommand{\includegraphicsInText}[1]{\includegraphicsVCentered{height=\baselineskip}{#1}}

%%% Same as includegraphicsInText but for arbitrary text
\newcommand{\useboxInText}[2][\baselineskip]{\ensuremath{\vcenter{\hbox{\resizebox{!}{#1}{\usebox{#2}}}}}}

%%% Takes the name of a macro (first argument) and defines it with the given body (second argument).
\newcommand{\defineCommandByName}[2]{\expandafter\newcommand\csname #1\endcsname{#2}}
%%% Same as \defineCommandByName but inlines the body. (Under the hood "edef" instead of "newcommand" is used.)
\newcommand{\defineCommandByNameInlined}[2]{\expandafter\edef\csname #1\endcsname{#2}}
%%% Same as \defineCommandByName but uses renewcommand instead of newcommand.
\newcommand{\redefineCommandByName}[2]{\expandafter\renewcommand\csname #1\endcsname{#2}}

%%% Same as \defineCommandByName but only defines the command it is not already defined.
\newcommand{\defineCommandByNameIfUndefined}[2]{\ifcsname #1\endcsname\else\defineCommandByName{#1}{#2}\fi}

%%% Takes the name of a macro and runs the macro with that name without arguments.
%%% Make sure that the macro with the given name does not take any arguments.
%%% For example \runCommandByName{dots} will evaluate to \dots.
\newcommand{\runCommandByName}[1]{\csname #1\endcsname}

\usepackage{tikz}
\usetikzlibrary{shapes.symbols}
\usetikzlibrary{arrows.meta}
\usetikzlibrary{fit}
\usetikzlibrary{shapes,decorations.text,shadows,fadings,positioning,overlay-beamer-styles}
\usetikzlibrary{calc}
\usetikzlibrary{shapes} % for strike out
\usetikzlibrary{tikzmark}
\usepackage{tikzscale} % for \includegraphics with tikz files
\usepackage{newunicodechar}
\newunicodechar{∧}{\ensuremath{\wedge}}
\newunicodechar{∨}{\ensuremath{\vee}}
\newunicodechar{¬}{\ensuremath{\neg}}
\tikzstyle{codesnippetnode} = [text width=1.5cm]
%\newcommand{\codeSnippetFromFile}[3][]{}
\usepackage{minted}
\newcommand{\codeSnippetFromFile}[3][]{\inputminted[%
	style=colorful,
	% autogobble,
	% stripnl,
	gobble=0,
	tabsize=4,
	numbersep=1ex,
	rulecolor=gray,
	%            linenos,
	fontsize=\large\selectfont,
	numberblanklines=false, 
	% framesep=1ex,
	%            frame=lines,
	baselinestretch=1.1,
	#1]{#2}{#3}}
\newcommand{\latexphddir}{LaTeXPhD}
\input{LaTeXPhD/VariationDiffs.tex}
\input{LaTeXPhD/vedits/tikz/fse22visualabstract/defines.tex}
\input{LaTeXPhD/vedits/tikz/fse22visualabstract/boxes.tex}

\usetikzlibrary{calc}
\usetikzlibrary{intersections}

\usepackage{hyperref}

\usepackage{xcolor}
\definecolor{uulmBlue}{RGB}{125,154,170}
\definecolor{csRed}{HTML}{A32638}


%\usepackage[all]{hypcap}
\newcommand*\numcircledtikz[1]{\tikz[baseline=(char.base)]{
		\node[circle,draw,double,inner sep=1.2pt] (char) {#1};}}




















