\chapter{Hintergrund}
\section{C-Präprozessor}

C-Präprozessor ist ein Tool, das den Quellcode vor dem Kompilieren manipuliert~\cite{ABKS13}. Dieses Tool bietet Möglichkeiten zur bedingte Kompilierung, zur  Dateieinbindung und zur Erstellung lexikalische Makros~\cite{ABKS13}. Eine C-Präprozessor-Direktive beginnt mit \# und geht bis zum ersten Whihespace-Zeichen weiter, optional kann nach der Direktive Argument im Rest der Zeile stehen. Der C-Präprozessor hat solche Anweisungen wie, \#include zum Einbinden von Dateien um zum Beispiel Header-Dateien wiederzuverwenden. Wie das Aussehen kann ist in der Abbildung 2.1 Zeile 1 zu sehen(Abb.2.1 Z1). Mit den Anweisungen \#if(Abb.2.1 Z6), \#else(Abb.2.1 Z10), \#elif(Abb.2.1 Z8), \#ifdef(Abb.2.1 Z18), \#ifndef(Abb.2.1 Z2), und \#endif(Abb.2.1 Z4) wird die bedingte Kompilierung erzeugt. Dabei funktionieren \#if, \#else, \#elif, und \#endif vergleichbar mit dem was man aus Programmiersprachen und Pseudocode gewohnt ist. \#ifdef ist ähnlich zu \#if, wird aber nur dann Wahr wenn der drauf folgender Makros definiert ist. \#ifndef ist die Negation von \#ifdef. Die Makros werden durch die Anweisung \#define(Abb.2.1 Z3) erstellt. Der Präprozessor ersetzt dann wehrend seiner Arbeit, dem Makronamen durch seine Definition. Während dieser Arbeit kann ein Makros definiert, umdefiniert und undefiniert werden. Der C-Präprozessor hat noch weitere Anweisungen, auf die wir nicht weiter eingehen. Der C-Präprozessor kann in anderen Programmiersprachen verwendet werden, wenn diese Sprachen syntaktisch ähnlich zu C sind. Beispiel für solchen Sprachen sind C++, Assemblersprachen, Fortran und Java. Der Grund dafür ist, dass der C-Präprozessor ist unabhängig von der zugrundeliegenden Programmiersprache ist. Eine so ähnliche Vorverarbeitungsmöglichkeit ist in vielen anderen Programmiersprachumgebungen.\\


\lstdefinestyle{customc}{
	belowcaptionskip=1\baselineskip,
	breaklines=true,
	frame=L,
	xleftmargin=\parindent,
	language=C,
	numbers=left,
	numbersep=6pt,
	showstringspaces=false,
	basicstyle=\footnotesize\ttfamily,
	keywordstyle=\bfseries\color{green!40!black},
	commentstyle=\itshape\color{purple!40!black},
	identifierstyle=\color{blue},
	stringstyle=\color{orange},
}

\lstset{escapechar=@,style=customc}
\begin{figure}[H]
\hspace*{3cm}
\begin{minipage}{\textwidth}
\begin{lstlisting}
#include <stdio.h>
#ifndef N
#define N 10
#endif
	
#if N > 10
#define A "^-^"
#elif N == 10
#define A ";)"
#else
#define A ":("
#endif
	
	int main()
	{
		int i;
		puts("Hello world!");
#ifdef N
		for (i = 0; i < N; i++)
		{
			puts(A);
		}
#endif
		
		return 0;
	}
\end{lstlisting}
\end{minipage}
%\lstinputlisting[caption=Scheduler, style=customc]{chapters/hintergrund/main.c}
\caption{Beispiel für C Code mit Präprozessor Anweisungen}
\end{figure}


\section{Variabilität Umsetzung mit C-Präprozessor}
Der C-Präprozessor ist eine Möglichkeit, Variabilität zu erzeugen~\cite{ABKS13}. Um die Variabilität mithilfe des C-Präprozessors zu erzeugen, brauchen wir dessen Möglichkeit zur bedingten Kompilierung~\cite{ABKS13}. Dabei können beliebige Aussageformeln über Features im Quellcode mit den C-Präprozessor-Anweisungen \#if, \#ifdef und, \#ifndef abgebildet werden~\cite{BTS+:ESECFSE22} .\\















