\chapter{Fazit und Zukünftige Arbeiten}

Der Unparser sollte Variation-Trees und Variation-Diffs in C-Präprozessor-Annotierten Code und textbasierte Diffs überführen, dieser ist aber sowohl nicht in DiffDetective als auch anders wo gegeben. Wir entwickelten einen Algorithmus zum Unparsen von Variation-Trees und ein Vorgehen zum Unparsen von Variation-Diffs. Um das zu bewerkstelligen haben wir uns mit Definitionen von Variation-Trees und Variation-Diffs beschäftigt und die gegebenen Definitionen für unsere Zwecke erweitert. Damit wir den Definitionen gerecht werden haben wir uns mit dem Parser beschäftigt, um den zu erweitern. Den ausgearbeiteten Algorithmus zum Unparsen von Variation-Trees und das Vorgehen zum Unparsen von Variation-Diffs haben wir vorgestellt und in DiffDetective implementiert. Damit wir zeigen können, das unser Unparser korrekt ist, haben wir Korrektheitskriterien ausgearbeitet und eine Auswertung auf denen durchgeführt.\\

Anhand der durchgeführten Auswertung auf den Datensätzen \texttt{Vim}, \texttt{sylpheed} und \\ \texttt{berkeley-db-libdb} haben wir erkenntlich gemacht für welche Parser-Optionen wie korrekt unser Unparser funktioniert. Die Auswertung der Korrektheit des Unparsers wurde, auf den von uns definierten Korrektheitskriterien durchgeführt. Für die Parser-Optionen, für die unser Unparser für das Unparsen von Variation-Trees entwickelt wurde, funktioniert der korrekt. Dies haben wir an 98802 Testfällen gesehen. Der Unparser für das Unparsen von Variation-Diffs funktioniert leiden nicht so gut, 11 der 49401 der Testfälle konnten nicht ungeparst werden, welches weniger als 0,03$\%$ aller Testfälle darstellt.\\

Dank dem von uns ausgearbeiteten Unparsers ist es möglich geworden Verfahren durchzuführen, welche zuerst einen Variation-Tree bzw. Variation-Diff erstellen, den bearbeiten und dann dies als C-Präprozessor-Annotierten Code bzw. textbasierten Diff brauchen für weitere Bearbeitungen. Zu solchen Verfahren gehört Mutation-Tests für Variabilität. Damit sind die Möglichkeiten für Analysen der Variabilität, für welche die Entwickler DiffDetective verwenden können, größer geworden.\\


Wir sehen zwei mögliche Punkte für weitere Forschung. Als erster wäre es die Ausarbeitung und Implementierung eines Algorithmus zum Unparsen von Variation-Diffs. Unser Algorithmus ist nur für das Unparsen von Variation-Trees ausgelegt und nicht Variation-Diffs. Damit auch Variation-Diffs ungaparst werden können, reduzieren wie das Problem auf das Unparsen von Variation-Trees. Zwar funktioniert unser Vorgehen, aber wie die Auswertung zeigt, funktioniert unser Vorgehen nicht immer. Ein eigener Algorithmus zum Unparsen von Variation-Diffs könnte das vielleicht beseitigen, dazu könnten wir damit die Abhängigkeit von dem Differ-Algorithmus loswerden und die Korrektheit steigern. Als ein wichtiges Problem zum Entwickeln eines Algorithmus zum Unparsen von Variation-Diffs finden wir es, den Variation-Diff, welcher ein azyklischer Graph ist, zu entnehmen in welcher Reihenfolge die Knoten des Graphen besucht werden sollen. Wenn man dieses Problem überwindet, welches uns nicht gelungen ist, sollte man den Algorithmus zum Unparsen von Variation-Diffs näher sein. Als zweiter Punkt wäre, dass man sowohl Parser als den Unparser auf die Arbeit mit anderen Präprozessoren erweitert. Zurzeit können sowohl Parser als auch der Unparser nur mit den C-Präprozessor arbeiten. Ein möglicher Kandidat wäre der Java-Präprozessor, bei denen die Anweisungen an den C-Präprozessor angelehnt sind. 












