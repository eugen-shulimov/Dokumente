\chapter{Fazit und Zukünftige Arbeiten}



Wir sehen zwei mögliche Punkte für weitere Forschung. Als erster wäre es die Ausarbeitung und Implementierung eines Algorithmus zum Unparsen von Variation-Diffs. Unser Algorithmus ist nur für das Unparsen von Variation-Trees ausgelegt und nicht Variation-Diffs. Damit auch Variation-Diffs ungaparst werden können, reduzieren wie das Problem auf das Unparsen von Variation-Trees. Wir sahen als größte Hürde den Variation-Diff, welcher ein azyklischer Graph ist, zu entnehmen in welcher Reihenfolge die Knoten des Graphen besucht werden sollen. Wenn man diese Hürde überwindet, welches uns nicht gelungen ist, sollte man den Algorithmus zum Unparsen von Variation-Diffs näher sein. Als zweiter Punkt wäre, dass man sowohl Parser als den Unparser auf die Arbeit mit anderen Präprozessoren erweitert. Zur Zeit können sowohl Parser als auch der Unparser nur mit den C-Präprozessor arbeiten. Ein möglicher Kandidat wäre der Java-Präprozessor, bei dehnen die Anweisungen an den C-Präprozessor angelehnt sind. 












