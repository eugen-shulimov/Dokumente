Es gibt verschiedene Möglichkeiten Variabilität in einem Projekt umzusetzen. Einer dieser Möglichkeiten ist die Nutzung von C-Präprozessor-Annotationen. Dies gestattet uns Variabilität umzusetzen. Es gibt eine Reihe an Analysen und Forschungsarbeiten, Entwickler bei der Umsetzung der Variabilität und deren Analyse zu unterstützen. Dazu werden Tools wie DiffDetective verwendet. Zwar hat DiffDetective einen Parser, aber keinen Unparser. Variation-Trees und Variation-Diffs sind zwei zentralen Datenstrukturen in DiffDetective, um Präprozessorvariabilität und Änderungen daran darzustellen. In dieser Arbeit präsentieren wir einen Unparser für Variation-Trees und Variation-Diffs. Wir haben diesen Algorithmus in DiffDetective implementiert und ein einem großen Datensatz validiert. Die von uns gewählten Datensätze sind Vim, sylpheed, gcc und berkeley-db-libdb. Für die Validierung wurde von uns mehrere Korrektheitskriterien für unseren Unparser ausgearbeitet. Damit man feststellen kann, ob ein Unparser syntaktisch oder semantisch korrekt arbeitet.